\newpage
\clearpage

\section{Investigation}
\label{sec:investigation}

To investigate such reciprocal sharing opportunities, we obtain a \wifi{} scan
result dataset from \PhoneLab{} (\S\ref{subsec:phonelab}). Extensive analysis
reveals that many users can benefit from accessing even one extra \wifi{} AP
(\S\ref{subsec:better}) and reciprocal relationships do exist
(\S\ref{subsec:reciprocal}).

\subsection{PhoneLab \wifi{} Dataset}
\label{subsec:phonelab}

\begin{table}[t]
  \begin{tabularx}{\columnwidth}{Xr}
    \toprule
    Begin & 11/7/2014 \\ 
    End & 4/3/2015 \\ 
    Duration (Days) & 147 \\ \midrule
    Participants & 254 \\
    Device Type & Nexus~5 \\ \midrule
    Scans & \num{21192417} \\
    Observed APs & \num{1197522} \\
    Used APs & \num{15668} \\ \midrule
    \wifi{} Sessions & \num{466032} \\
    Total Connection Time (Days) & \num{34721} \\
    \bottomrule
  \end{tabularx}
  \caption{\textbf{\PhoneLab{} \wifi{} Dataset Summary.} Used APs refers to the subset of
  total APs that were used by the devices participating in the study.}
  \label{tab:summary}
\end{table}

\PhoneLab{}\cite{phonelab-sensemine13} is a public smartphone platform testbed
operated at the University at Buffalo. Several hundreds of participants carry
instrumented Nexus 5 smartphones as their primary device. In particular, the
platform was modified to log each \wifi{} scan results as well as connection
events naturally generated by the Android system. Note that such information can
also be logged at application level given appropriate permissions.
Table~\ref{tab:summary} summarizes the \PhoneLab{} \wifi{} dataset.

\wifi{} scan result represents the device's network visibility. Multiple entries
are reported in one scan result, each corresponding to one \wifi{} Access Point
(AP) the device observed. The content of one entry includes: (1) scan timestamp,
(2) AP SSID and BSSID, (3) AP channel and (4) RSSI.

\subsection{Home AP Detection}
\label{subsec:homeap}

We first develop several heuristics to identify the home AP for each device in
the dataset. The intuition is that the devices are most likely connected to
home AP at night. More specifically, we first identify home AP candidates for
each device by looking at the \wifi{} sessions happened during 12~AM and 4~AM.
We then filter out candidate APs that provides fewer than 100 sessions or less
than 10 hours of total connection time.

After applying the above heuristics, the home AP information of 123 devices are
identified, including 118 unique SSIDs and 126 unique BSSIDs. Some factory
preset SSIDs, such as \texttt{linksys}, are used by multiple home APs and
certain devices have more than one home BSSIDs due to AP replacement during the
data collection period.

\subsection{Better AP}
\label{subsec:better}

We then look into these two questions: (1) When the device is connected to home
APs, how often does it receive a better signal from neighbors' APs that it does
not have access? and (2) Is there a particular neighbor AP that consistently
provides better signals than the device's home AP?

To answer the first question, we inspect scan results that are reported during
\wifi{} sessions with home APs. For each such scan result, we identify the
currently associated home AP, $AP_{home}$, and the AP with best RSSI among all
reported APs, denoted as $AP_{best}$. We are particularly interested in the
\textit{sub-optimal} cases where: (1) $AP_{home} \neq AP_{best}$; (2) the device
never connects to $AP_{best}$ and (3) the RSSI of $AP_{best}$ is
better than $AP_{home}$ by a threshold (5dBm in our analysis). Such cases indicate that the
device could potentially improve its \wifi{} performance by connecting to
neighbors' APs with better signal yet it does not have access to. Note that here we
only consider RSSI in determining the \textit{optimal} AP and it is well studied that
RSSI does not directly translate to \wifi{} performance, which we will discuss
in Section~\ref{sec:challenges}. Also note that the cases when the device are not
connecting to APs with best signal due to bad roaming strategies are not
interesting in the context of this paper, and are excluded by the second condition.

\begin{figure}[t]
  \centering
  \includegraphics[width=\columnwidth]{./figures/HomeAPSessionRSSI.pdf}
  \caption{\textbf{CDF of Sub-Optimal Connection Time.}}
  \label{fig:suboptimal}
\end{figure}
\begin{figure}[t]
  \centering
  \includegraphics[width=\columnwidth]{./figures/BetterNeighborAPFigure.pdf}
  \caption{\textbf{CDF of Dominant AP Fraction.}}
  \label{fig:dominantap}
\end{figure}

Figure~\ref{fig:suboptimal} shows the CDF of percentage of sub-optimal time of
the 123 devices. We make several observations. First, for 80\% of the devices,
their home APs usually provides best signal (sub-optimal percentage less than
20\%). This result is not particularly surprising considering that home APs are
usually carefully positioned to try to provide good coverage. Second, we notice
that for certain number of devices (10\%), their home APs failed to provide best
signal for more than 40\% of the time, suggesting that their users may benefit
from sharing the \wifi{} access of their neighbor APs.

To see whether there is a particular neighbor AP that consistently provides
better signal, we first calculate the distribution of $AP_{best}$ in sub-optimal
cases, and then calculate the fraction of the dominant AP which provides better
signal more often than others.

Figure~\ref{fig:dominantap} shows the CDF of dominant AP fraction. For 50\% of
the devices, one particular neighbor AP contributes to more than 60\% of
sub-optimal connection time, implying that even by sharing \wifi{} access of one
neighbor AP, these device's sub-optimal connection time can be reduced by more
than 60\%.

\subsection{Reciprocal Sharing}
\label{subsec:reciprocal}

Finally, we investigate the cases where two devices can obtain better signals
from each other's home AP, i.e., reciprocal sharing opportunity.

\newpage
\clearpage

