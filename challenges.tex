\section{Challenges}
\label{sec:challenges}

We identify several challenges posed by such a \wifi{} sharing system as
\wisefi{} and discuss possible solutions for certain challenges.

\subsection{Dynamic AP Configuration}
\label{subsec:config}

Dynamic AP configuration is crucial in easy and flexible \wifi{} sharing. For
instance, in Section~\ref{subsec:sharing}, \wifi{} sharing without virtual guest
network support is challenging: simply sharing the \wifi{} credential of user's
home AP to other \wisefi{} users is not only dangerous, but also making it
difficult to revoke the access in the future. In worst case scenario, a user may
be forced to change the home AP password and reconfigure the \wifi{} credential
on all his/her devices just to revoke the access of the other \wisefi{} user.
Although most commodity APs support client MAC black or white list feature,
configuring them properly is difficult for average users. Furthermore, the
sharing relationship should be built between users instead of devices: once the
sharing is established, one user should be able to connect any of his/her
devices, not only the smartphone, to the other user's home AP. Even the system
can directly share each other's \wifi{} credential, manually configuring it on
all devices is cumbersome. Similarity, in Section~\ref{subsec:monitoring}, when
monitoring the network usage to ensure reciprocity, it is difficult to draw
conclusions simply from the measurements of the smartphone alone.

To overcome these challenges, we envision a programmable \wifi{} AP
configuration API with three simple interfaces:
\texttt{getAuthenticatedClients}, \texttt{setWhiteList}, and
\texttt{getWhiteListClients}. \texttt{getAuthenticatedClients} simply returns
all the MAC addresses of clients that are associated with the AP through normal
authentication. \texttt{setWhiteList} sets a list of white list MAC
addresses that the AP must accept their association requests regardless of the
possible authentication errors (e.g., due to incorrect \wifi{} password).
Finally, \texttt{getWhiteListClients} returns the MAC addresses of clients that
associate with the AP through white list mechanism. For
security concerns, the AP will only accept these requests when they are sent by
authenticated devices. The API can be implemented on top of
existing SNMP protocols, or be provided in form of RESTful API through the HTTP
server that is already integrated in most commodity APs.

With the help of these configuration APIs, the \wifi{} sharing can work as
follows. Suppose the \wisefi{} system has discovered the reciprocal sharing opportunity
between Alice and Bob. First, the \wisefi{} app on Bob's smartphone (which
is associated with Bob's home AP through proper authentication) sends a
\texttt{getAuthenticatedClients} request to Bob's home AP, retrieving the MAC
addresses of all Bob's devices. These MAC addresses are uploaded to
\wisefi{} server and then forwarded to the \wisefi{} app on Alice's smartphone,
which sends a \texttt{setWhiteList} request to Alice's home
AP to add all Bob's devices to its white list. At this point, Bob can connect any of
his device to Alice's home AP using a dummy password. Later on, when the
reciprocal sharing opportunity no long exists, the \wisefi{} server can instruct
Alice's smartphone to perform another \texttt{setWhiteList} request to revoke
Bob's access to Alice's home AP by removing the MAC addresses of Bob's devices
from the white list.

There are several advantages of this sharing approach. First, note that
throughout the grant and revoke process, the \wifi{} credential of Alice's home
AP is not shared with Bob or the \wisefi{} server, thus remains confidential.
Second, revoking access of other \wisefi{} users simply requires a
\texttt{setWhiteList} request, without needing to change the user's home AP
\wifi{} credential.  Furthermore, the \wisefi{} app can list other \wiseif{}
users who are in a reciprocal sharing relationship and provide interfaces to let
user manually revoke access of other users. Finally, protection and isolation
can be enforced at the AP side by differentiating two type of clients:
authenticated clients (user's own devices) and while list clients (\wisefi{}
devices). Therefore, such sharing mechanism meets both the control and
protection goals in Section~\ref{subsec:sharing}.

Similarly, network usage monitoring can be achieved by periodic
\texttt{getWhiteListClients} requests sent by the \wisefi{} app to monitor which
other \wisefi{} devices are currently associated with the home AP. Long term
measurements can then be aggregated to ensure the reciprocity of sharing.

\subsection{Privacy and Security}


\subsection{Bootstrap and Incentives}

