\section{Challenges}
\label{sec:challenges}

\subsection{Credential Sharing}
\label{subsec:credential}

As discussed in Section~\ref{subsec:sharing}, \wifi{} sharing without virtual
guest network support is challenging. Simply sharing the \wifi{} credential of
user's home AP to other \wisefi{} users is not only dangerous, but also making
it difficult to revoke the access in the future. In worst case scenario, a user
may be forced to change the home AP password and reconfigure the \wifi{}
credential on all his/her devices just to revoke the access of the other
\wisefi{} user. Although most commodity APs support client MAC black or white
list feature, configuring them correctly is difficult for average users.

On other hand, the sharing relationship should be built between users instead of
devices. For instance, once the sharing is established, one user should be able
to connect any of his/her devices, not only the smartphone, to the other user's
home AP. Even the system can directly share the \wifi{} credential, configuring it
on all devices is cumbersome.

To overcome this challenge, we envision a programmable \wifi{} AP configuration
API with two simple interfaces: \texttt{getAuthenticatedClients} and
\texttt{setWhiteList}.  \texttt{getAuthenticatedClients} simply returns all the
MAC addresses of clients that are associated with the AP through normal
authentication. And \texttt{setWhiteList} sets a list of white list MAC addresses
that the AP must accept their association requests regardless of the possible
authentication errors (e.g., due to incorrect \wifi{} password). For security
concerns, the AP accepts these two requests only if they are sent by devices that
are correctly authenticated. The API can be implemented on top of existing SNMP
protocols, or be provided through RESTful API through the HTTP server that
is already integrated in most commodity APs.

With the help of these two configuration APIs, the \wifi{} sharing can work as
follows. Suppose the system has discovered the reciprocal sharing opportunity
between user Alice and Bob. First, the \wisefi{} app on Bob's smartphone (which
is associated with Bob's home AP through correct authentication) performs a
\texttt{getAuthenticatedClients} request from Bob's home AP, to get the MAC
addresses of all Bob's devices. The list of MAC addresses is then uploaded to
\wisefi{} server and forwarded to the \wisefi{} app on Alice's smartphone.
Alice's smartphone can then send \texttt{setWhiteList} request to Alice's home
AP to grant access to all Bob's devices. At this point, Bob can connect any of
his device to Alice's home AP using a dummy password. Later on, when the
reciprocal sharing opportunity no long exists, the \wisefi{} server can instruct
Alice's smartphone to perform another \texttt{setWhiteList} request to revoke
Bob's access to Alice's home AP by removing the MAC addresses of Bob's devices
from the white list.

There are several advantages of this sharing approach. First, note that in both
grant and revoke process, the \wifi{} credential of Alice's home AP is not
shared with Bob or the \wisefi{} server, thus remains confidential. Second,
revoking access of other \wisefi{} users simply requires a \texttt{setWhiteList}
request, without needing to change the user's home AP \wifi{} credential.
Furthermore, the \wisefi{} app can list other \wiseif{} users who are in a
reciprocal sharing relationship and provide interfaces to let user manually
revoke access of other users. Finally, protection and isolation can be enforced
at the AP side by differentiating two type of clients: authenticated clients
(user's own devices) and while list clients (\wisefi{} devices). Therefore, such
sharing mechanism meets both the control and protection goals posed in
Section~\ref{subsec:sharing}.

\subsection{Bootstrap and Incentives}

