\begin{abstract}

Widespread deployment of private home \wifi{} access points (APs) can
result in uncoordinated and overlapping wireless networks that compete with
each other for limited bandwidth. We expect this suboptimal arrangement to
only get worse, particularly in the dense urban environments that house
an increasing fraction of the world's population. Broadband penetration and
the demand for high-speed \wifi{} throughout the home will lead to more
private APs, which will generate more interference for neighboring
networks, resulting in even more private APs and additional interference,
and so on.

In this paper we investigate whether we can prevent this vicious cycle by
using \textit{reciprocal \wifi{} sharing} to make better use of existing
private home APs. We define reciprocal \wifi{} sharing as cases where two
users both improve their network performance by connecting to each other's
overlapping private \wifi{} networks. Compared to previous approaches that
attempted to use private APs to create large-scale open-access \wifi{}
networks, reciprocal \wifi{} sharing relationships more closely mirror
existing human relationships and can be maintained without elaborate
reputation mechanisms.

To evaluate the potential for reciprocal \wifi{} sharing, we analyze 21~M
\wifi{} scans collected from 254~smartphones over 5~months. Our results
show that even in a sparsely-populated suburban area, reciprocal \wifi{}
sharing can be beneficial. And surprisingly, we detected several reciprocal
\wifi{} sharing opportunities even within our tiny user sample. Motivated
by these results, we present the design of \wisefi{}, a system enabling
reciprocal \wifi{} sharing.

\end{abstract}
