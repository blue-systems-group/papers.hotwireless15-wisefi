\begin{abstract}
  Household \wifi{} network penetration keeps increasing to meet the demand of
  ubiquitous network access. Due to blockage and fading in signal propagation,
  however, \wifi{} APs can not provide uniform signal coverage within its
  vicinity. There are cases when a pair of users get better \wifi{} signal from
  each other's home AP at certain locations within their house, creating
  reciprocal sharing opportunities where they can improve their network
  performance by allowing each other to access their own private wifi{}
  networks. Such sharing relationship is \textit{immediate} between two parties
  and \textit{stable} over time, making it interesting to explore and practical
  to establish.

  In this paper, we investigate reciprocal \wifi{} sharing opportunities in real
  life scenarios through extensive analysis of the 21 million scan results
  collected from 254 smartphones for 5 months. Reciprocal relationships are
  revealed despite the spatial sparsity of the dataset. Inspired by the results,
  we present the design of \wisefi{}, a system that detects and enables
  reciprocal \wifi{} sharing.
\end{abstract}
